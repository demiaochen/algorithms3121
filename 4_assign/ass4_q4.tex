\documentclass[12pt]{article}

% Language setting
% Replace `english' with e.g. `spanish' to change the document language
\usepackage[english]{babel}

% Set page size and margins
% Replace `letterpaper' with`a4paper' for UK/EU standard size
\usepackage[letterpaper,top=2cm,bottom=2cm,left=2.8cm,right=2.8cm,marginparwidth=1.75cm]{geometry}

% Useful packages
\usepackage{amsmath, amssymb}
\usepackage{graphicx} 
\usepackage[colorlinks=true, allcolors=blue]{hyperref}
\newcommand*{\Perm}[2]{{}^{#1}\!P_{#2}}%
\newcommand*{\Comb}[2]{{}^{#1}C_{#2}}%


\title{COMP3121 Assignment 4 - Q4}
\author{Demiao Chen z5289988}

\begin{document}
\maketitle
\addcontentsline{toc}{section}{Acknowledgement}

\subsection*{Answer}
We initialise a 2D array board with size $n \times n$, 
then for each white shop $s_i$,
based on its position $(a_i, b_i)$, we mark board[$a_i$][$b_i$] and all 
entries on its diagonal directions as ``attacked".\\
Model this as a max flow problem. We construct a 
bipartite graph with vertices on the left side 
representing $n$ rows in the board,  denoted as $r_i$,
and vertices on the left side representing $n$ columns of 
the board, denoted as $l_i$.
Introduce a super source and a super sink.
Connect each $r_i$ with the super source 
by a directed edge of capacity equal to 1.
Connect each $l_i$ with the super
sink with a directed edge of capacity equal to 1.
For constructing the edges between $r_i$ and $l_i$, we check 
the array we made, if board[$r_i$][$l_i$] is not marked as 
``attacked" (means the position is not under attacked by white bishop), 
we construct a edge from $r_i$ to $l_i$, with capacity equal to 
1. \\
Now, the largest number of black rooks we can place is the max flow 
of our constructed graph. We run Edmonds-Karp algorithm on our
graph to get the residual graph. In the final residual graph, 
the sum of the weight of the incoming edges towards source equals to 
the max flow, which is the largest number we are looking for.\\\\
Time complexity: constructing the 2D array board takes $O(kn)$, $k \le n^2$.
Since the time complexity of 
Edmonds-Karp algorithm is $O( | V || E | ^2 )$, 
where $|E|$ is the number of edges, $| V |$ is the number of 
vertices, we have maximum $2n + n^2$ edges, 
and $2n + 2$ vertices in the constructed graph, 
hence the time complexity to get the  max flow is $O((2n + 2) \cdot (2n+n^2)^2) = O(n^5)$.
Therefore the overall time complexity of our algorithm is $O(kn) + O(n^5) = O(n^5)$.





\end{document}

