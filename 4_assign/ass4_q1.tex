\documentclass[12pt]{article}

% Language setting
% Replace `english' with e.g. `spanish' to change the document language
\usepackage[english]{babel}

% Set page size and margins
% Replace `letterpaper' with`a4paper' for UK/EU standard size
\usepackage[letterpaper,top=2cm,bottom=2cm,left=2.8cm,right=2.8cm,marginparwidth=1.75cm]{geometry}

% Useful packages
\usepackage{amsmath, amssymb}
\usepackage{graphicx} 
\usepackage[colorlinks=true, allcolors=blue]{hyperref}
\newcommand*{\Perm}[2]{{}^{#1}\!P_{#2}}%
\newcommand*{\Comb}[2]{{}^{#1}C_{#2}}%


\title{COMP3121 Assignment 4 - Q1}
\author{Demiao Chen z5289988}

\begin{document}
\maketitle
\addcontentsline{toc}{section}{Acknowledgement}

\subsection*{Answer}
We model this as a Max Flow - Min Cut problem. We first construct a 
a flow network as a directed graph where computer 1 is the source,
computer $N$ is the sink, and the rest computers
are vertices between source and sink. The capacity of each edge between
two vertices will be equal the cost of removing its corresponding link
between two computers.\\
We now need to find a min cut in the graph, which produces two subsets 
of the graph $S$ and $T$, where $S \cup  T = V$, $S \cap T = \varnothing$ and
$s \in S$, $t \in T$. Removing the edges in min cut will make computer 1
and computer $N$ not connected, and the cost is minimum. So our goal 
is to find the min cut in the graph.\\
To find the minimum cut, we run Edmonds-Karp algorithm to our
flow network graph to get the residual graph. Then we take 
source as start vertex, run breadth first search in the residual graph.
Hence we get a set of vertices that is reachable from the source,
those are the vertices belong to $S$, and the rest vertices will belong 
to $T$. Then we find the minimum cut, that is the edges 
with forward direction in the original flow network graph
between set $S$ and set $T$. Therefore, the links 
we need to remove are the links between $S$ and $T$ with forward direction.\\\\
Time complexity: since the time complexity of Edmonds-Karp is $O(|V||E|^2)$, where $|V|$
is number of vertices, $|E|$ is the number of edges, and time complexity of breadth
first search algorithm is $O(|V|+|E|)$. It takes $O(|E|)$ 
to find all edges between two sets. We have
$M$ edges (links) and $N$ vertices (computers).
Therefore, the time complexity of our algorithm to find the links to remove is $O(NM^2 + N + 2M) = O(NM^2)$



\end{document}
