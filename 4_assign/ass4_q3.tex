\documentclass[12pt]{article}

% Language setting
% Replace `english' with e.g. `spanish' to change the document language
\usepackage[english]{babel}

% Set page size and margins
% Replace `letterpaper' with`a4paper' for UK/EU standard size
\usepackage[letterpaper,top=2cm,bottom=2cm,left=2.8cm,right=2.8cm,marginparwidth=1.75cm]{geometry}

% Useful packages
\usepackage{amsmath, amssymb}
\usepackage{graphicx} 
\usepackage[colorlinks=true, allcolors=blue]{hyperref}
\newcommand*{\Perm}[2]{{}^{#1}\!P_{#2}}%
\newcommand*{\Comb}[2]{{}^{#1}C_{#2}}%


\title{COMP3121 Assignment 4 - Q3}
\author{Demiao Chen z5289988}

\begin{document}
\maketitle
\addcontentsline{toc}{section}{Acknowledgement}

\subsection*{Answer}
We model this as a max flow problem, with vertex capacities. We first construct a 
flow network as a directed graph where square 1 is the source, square $n$ is the 
sink. Each rest square $i$ will be split into two vertices as $v_{iin}$ and $v_{iout}$,
each $v_{iin}$ has only one outgoing edge which towards to $v_{iout}$, the capacity of 
the edge between $v_{iin}$ and $v_{iout}$ equals to $A[i]$. Each vertex $v_{iout}$ and source
has directed edges towards to $v_{(i+1)in}$,  $v_{(i+2)in}$ ...  $v_{(i+k)in}$, with
capacity equal to infinity.\\
To find the largest number of children who can successfully complete the game
is to find the max flow in our constructed graph. We now run Edmonds-Karp algorithm
on our graph, in the final residual graph, the sum of the weight of the incoming 
edges towards source is the largest number we are looking for.\\\\
Time complexity: since the time complexity of Edmonds-Karp algorithm is 
$O(|V||E|^2)$, where $|E|$ is the number of edges, $|V|$ is the number of 
vertices, we have $k n + n$ edges, $2n + 2$ vertices in the graph, 
hence the time complexity is $O((2n+2)\cdot(kn+n)^2)$ = $O(k^2n^3)$.




\end{document}

