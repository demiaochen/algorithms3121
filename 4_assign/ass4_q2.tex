\documentclass[12pt]{article}

% Language setting
% Replace `english' with e.g. `spanish' to change the document language
\usepackage[english]{babel}

% Set page size and margins
% Replace `letterpaper' with`a4paper' for UK/EU standard size
\usepackage[letterpaper,top=2cm,bottom=2cm,left=2.8cm,right=2.8cm,marginparwidth=1.75cm]{geometry}

% Useful packages
\usepackage{amsmath, amssymb}
\usepackage{graphicx} 
\usepackage[colorlinks=true, allcolors=blue]{hyperref}
\newcommand*{\Perm}[2]{{}^{#1}\!P_{#2}}%
\newcommand*{\Comb}[2]{{}^{#1}C_{#2}}%


\title{COMP3121 Assignment 4 - Q2}
\author{Demiao Chen z5289988}

\begin{document}
\maketitle
\addcontentsline{toc}{section}{Acknowledgement}

\subsection*{Answer}
We model this as a max flow problem. 
Make a bipartite graph with vertices on the left side 
representing warehouse and vertices on the right 
side representing shops. Introduce a super source and a super sink.
Connect each warehouse with the super source by a directed edge 
of capacity equal to 1 to represent if a truck has been sent in the shop. 
Connect each shop with the super sink with a directed edge of 
capacity equal to 1 to represent if a shop has been supplied. \\
We now sort all roads $r_i$ in terms of its time taken to drive $d_i$ to form an 
increasing sequence by merge sort. Then we take an $i$ between 1 and the 
number of roads $m$ in a binary search way, that is we first take
$i = \frac{m}{2}$, then we construct edges in our graph by adding 
directed edges point to shop vertices to represent roads
from $r_1$ to $r_i$ in the graph to connect corresponding 
warehouse and shop vertices, with capacity equal to 1.
Then we run 
Edmonds-Karp algorithm on our graph to get the residual 
graph, and get the max flow from the final residual graph. 
The max flow in our graph represent the maximum shops we can 
supply in the current constructed graph.
If max flow equals number of shops $n$, means all shops can be 
supplied, we take a smaller $i$ in binary search way and remove all edges to repeat
above process, 
else we take a bigger $i$ in binary search way and remove all edges to repeat
above process, 
until max flow of the graph equals $n$ and $d_i$ is minimised.\\
After finishing the above process, we have get the smallest $d_i$ and 
its corresponding flow network graph,
hence we run Edmonds-Karp algorithm in the 
final constructed graph to get the residual 
graph. In the final residual graph, we output 
each warehouse vertex and its connected edge (road) 
with direction towards the warehouse and
its connected shop vertex, which means this arrangement: send truck from the vertex
represented warehouse
to the vertex represented shop through the edge represented road.\\\\
%%
Time complexity: We used binary search to get $i$ from $i = 1$ to $i = m$, and each constructed graph we 
run Edmonds-Karp algorithm on it. As we have $n$ ware houses, $m$ roads,
there are $2n+2$ vertices in the graph in total, $m+2n$ edges (including 
edges between super source  and super sink), and time complexity of Edmonds-Karp algorithm
is $O(|V||E|^2)$,
hence the time complexity of our algorithm is $O((2n+2) \cdot (m+2n)^2 \cdot \log{m}) = O((m^2n + n^3 + mn^2) \cdot \log{m})$. 
It is obvious that $m \ge n$, so $O((m^2n + n^3 + mn^2) \cdot \log{m}) = O(m^2n \cdot \log{m})$.
The final result output process takes $O(n)$.\\\\
Therefore, the overall time complexity of our algorithm is $O(m^2n \cdot \log{m})+O(n)$ = 
$O(m^2n\cdot \log{m})$.




\end{document}

