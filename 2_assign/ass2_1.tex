\documentclass[12pt]{article}

% Language setting
% Replace `english' with e.g. `spanish' to change the document language
\usepackage[english]{babel}

% Set page size and margins
% Replace `letterpaper' with`a4paper' for UK/EU standard size
\usepackage[letterpaper,top=2cm,bottom=2cm,left=2.8cm,right=2.8cm,marginparwidth=1.75cm]{geometry}

% Useful packages
\usepackage{amsmath, amssymb}
\usepackage{graphicx} 
\usepackage[colorlinks=true, allcolors=blue]{hyperref}
\newcommand*{\Perm}[2]{{}^{#1}\!P_{#2}}%
\newcommand*{\Comb}[2]{{}^{#1}C_{#2}}%


\title{COMP3121 Assignment2 - Q1}
\author{Demiao Chen z5289988}

\begin{document}
\maketitle
\addcontentsline{toc}{section}{Acknowledgement}

\subsection*{Answer}
Always place on one CD until its remaining space is not enough to place the next song,
in this case, place the song on next CD and start placing on the new CD. 
\\ 
\\ 
E.g. At start, we place song 1 on $CD_1$, then place song 2 on $CD_2$ if and only if 
there is enough space remaining
to place it (i.e. $m-l_1 \geqslant l_2$), else place it on the next CD, which is CD 2.\\
If we are going to place song $k$ and the last song placed is on $CD_j$, with the first song placed on $CD_j$ is song $h$ ($h < k$),
we first decide if 
\begin{align}\label{eq:1} 
    m-\sum_{i = h}^{k}{l_i} \geqslant 0
\end{align}
If the condition hold, we place song $k$ on $CD_j$, otherwise on $CD_{j+1}$\\\\
We now need to prove that this method is optimal. In the above condition
to place song $k$, if (1) does not hold but place song $k$ on $CD_j$, this song cannot 
be placed completely on $CD_j$ since the space is not enough, and we cannot spill song 
across CDs. We cannot place song $k$ on previous CDs either as songs must be recorded in order.
So we must place the song on the next CD.\\
If (1) holds but place song $k$ on $CD_{j+1}$, this will not achieve the optimal solution 
since if song $k$ is the last song needs to be recorded, there will be more than one
CD needed compare to our method. Place on previous CDs will disorder.\\
Any possible violation of our algorithm will result in not satisfying the question or more 
CDs needed. Therefore, our method is optimal.\\\\
Time complexity: there are total of n songs to be placed, each time when placing a song we check 
if the reaming space on current CD is enough by using formula (1), which takes $O(1)$ as the time complexity of
formula (1) does not grow proportional to n, i.e. irrelative to n. So the total time complexity is $O(n*1) = O(n)$

\end{document}
