\documentclass[12pt]{article}

% Language setting
% Replace `english' with e.g. `spanish' to change the document language
\usepackage[english]{babel}

% Set page size and margins
% Replace `letterpaper' with`a4paper' for UK/EU standard size
\usepackage[letterpaper,top=2cm,bottom=2cm,left=2.8cm,right=2.8cm,marginparwidth=1.75cm]{geometry}

% Useful packages
\usepackage{amsmath, amssymb}
\usepackage{graphicx} 
\usepackage[colorlinks=true, allcolors=blue]{hyperref}
\newcommand*{\Perm}[2]{{}^{#1}\!P_{#2}}%
\newcommand*{\Comb}[2]{{}^{#1}C_{#2}}%


\title{COMP3121 Assignment 2 - Q4}
\author{Demiao Chen z5289988}

\begin{document}
\maketitle
\addcontentsline{toc}{section}{Acknowledgement}

\subsection*{Answer}
We can have a greedy method to solve the problem.
Loop array $A[i]$ from $i=1$ to $i=n-1$. We check the $i^{th}$ stack if $$
A[i] < A[i+1],
$$
if so, the $i^{th}$ stack and the $(i+1)^{th}$ stack are strictly increasing, so from the $1^{st}$ 
to the $(i+1)^{th}$ stack are strictly increasing in number of blocks, then go to the next index of array $A$ ;\\
if not, we have to satisfy that if there can be $x$ ($0 < x \leq i$) blocks moved from $i^{th}$ stack
to $(i+1)^{th}$ stack, such that the two following conditions hold:
\begin{align}\label{eq:1} 
    A[i+1] + x \geq  A[i] - x + 1\\
    A[i] - x \geq A[i-1]+ 1
\end{align}
that is to add a minimum integer $x$ to $A[i+1]$ that makes $A[i+1]$ strictly bigger than $A[i]$, but also keep $A[i]$
strictly bigger than $A[i-1]$ after subtract $x$.\\
Transform the two conditions (1) (2) we get the following
\begin{align}
   x \geq \frac{A[i]-A[i+i]+1}{2} \\
    x \leq A[i] - A[i-1] - 1
\end{align}
As $x$ must be an integer, we can get $x$ by 
\begin{align}
    x =\ulcorner\frac{A[i]-A[i+i]+1}{2}\urcorner 
\end{align}
that is get the celling of the right hand of (3).\\
Then we check if the $x$ from the result of (5) satisfies
\begin{align}
    x \leq i
\end{align}
and (4).\\
If either (4) or (6) $x$ does not satisfy, we conclude that such movements does not exist and 
end the algorithm.\\
Else we add the $x$ to $A[i+1]$ and go to the next index of array $A$.\\
If we finish the loop of array $A$, we conclude that such movements exists. Algorithm finished.\\\\
Time complexity: compute $x$ from (5) then do comparisons (4) and (6), or just one $
A[i] < A[i+1]
$ comparison take time $O(1)$, and the length of array
$A$ is $n$ so we need to do $n-1$ times such calculations. Therefore, the time complexity of
 the algorithm is $O((n-1)*1) = O(n)$ 


\end{document}