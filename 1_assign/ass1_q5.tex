\documentclass[12pt]{article}

% Language setting
% Replace `english' with e.g. `spanish' to change the document language
\usepackage[english]{babel}

% Set page size and margins
% Replace `letterpaper' with`a4paper' for UK/EU standard size
\usepackage[letterpaper,top=2cm,bottom=2cm,left=2.8cm,right=2.8cm,marginparwidth=1.75cm]{geometry}

% Useful packages
\usepackage{amsmath, amssymb}
\usepackage{graphicx} 
\usepackage[colorlinks=true, allcolors=blue]{hyperref}
\newcommand*{\Perm}[2]{{}^{#1}\!P_{#2}}%
\newcommand*{\Comb}[2]{{}^{#1}C_{#2}}%


\title{COMP3121 Assignment1 - Q5}
\author{Demiao Chen z5289988}

\begin{document}
\maketitle
\addcontentsline{toc}{section}{Acknowledgement}

\subsection*{Answer}
(a) $f(n) = O(g(n)).$\\
We show that eventually $\log_2{n} < \sqrt[10]{n}$. It is clear 
that both $\log_2{n}$ and $\sqrt[10]{n}$ are monotonically increasing function,
and $\sqrt[10]{n}$ eventually dominates $\log_2{n}$. By applying L’Hôpital rule
we get:
$$
\lim_{n \to \infty} \frac{\sqrt[10]{n}}{\log{n}} = 
\lim_{n \to \infty} \frac{(\sqrt[10]{n})^\prime}{(\log{n})'}=
\lim_{n \to \infty} \frac{\frac{1}{10}n^{-\frac{9}{10}}}{\frac{1}{n}}=
\lim_{n \to \infty} \frac{n^{1/10}}{10}=
\infty
$$
This proves that eventually $\log_2{n} < \sqrt[10]{n}$, hence 
$\log_2{n} = O(\sqrt[10]{n})$.
\\\\
% (b) f(n) = $\Theta(g(n))$. \\
% Take logarithm for both $n^n$ and $2^{n\log{n^2}}$ we get:
% $$
% \log{n^n} = n \log{n} = O(n\log{n})
% $$
% $$
% \log{2^{n\log{n^2}}} = 
% n\log{n^2}*\log{2} = 
% 2n\log{n}*\log{2}=
% 2\log{2} * n\log{n}
% = O(n\log{n})$$
% Hence, we conclude that $f(n)$ and $g(n)$ grow asymptotically the same, i.e. 
% $f(n) = \Theta(g(n))$.
(b) $f(n) = O(g(n))$. \\
By manipulating $g(n)$, we get $$
2^{n\log_2{n^2}} = 2^{\log_2{n^{2n}}} = n^{2n}.
$$
Then we get $$
\frac{g(n)}{f(n)} = \frac{n^{2n}}{n^n} = n^n.
$$
So it is clear that $f(n) < g(n)$ for $n>0$, $g(n)$ dominates $f(n)$.
Hence $f(n) = O(g(n))$.
\\\\
(c) $f(n) = \Theta (g(n))$. \\
As we assume $n$ is a positive integer, for any given $n$,
$\sin {\pi n} = 0$. \\
Hence, $$f(n) = n ^ {1 + \sin {\pi n}} = n ^ {1 + 0} = n = g(n).$$
Therefore, $f(n)$ and $g(n)$ have the same growth rate, so
$f(n) = O(g(n))$ and $g(n) = O(f(n))$ both hold, i.e. $f(n) = \Theta (g(n))$.

% Neither.\\
% Notice that when $n = \frac{1}{2} + 2k, k \in N$, $f(n)
% = n^2$, which grows much faster than $g(n) = n$. When $
% n = \frac{3}{2} + 2k, k \in N$, $f(n) = n^0 = 1$, so in this 
% case $g(n)$ grows much faster than $f(n)$. Thus, neither $
% f(n) = O(g(n))$ nor $
% g(n) = O(f(n))$ is satisfied.

\end{document}