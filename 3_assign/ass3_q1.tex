\documentclass[12pt]{article}

% Language setting
% Replace `english' with e.g. `spanish' to change the document language
\usepackage[english]{babel}

% Set page size and margins
% Replace `letterpaper' with`a4paper' for UK/EU standard size
\usepackage[letterpaper,top=2cm,bottom=2cm,left=2.8cm,right=2.8cm,marginparwidth=1.75cm]{geometry}

% Useful packages
\usepackage{amsmath, amssymb}
\usepackage{graphicx} 
\usepackage[colorlinks=true, allcolors=blue]{hyperref}
\newcommand*{\Perm}[2]{{}^{#1}\!P_{#2}}%
\newcommand*{\Comb}[2]{{}^{#1}C_{#2}}%


\title{COMP3121 Assignment 3 - Q1}
\author{Demiao Chen z5289988}

\begin{document}
\maketitle
\addcontentsline{toc}{section}{Acknowledgement}
Acknowledgement: the answer is inspired from tutorial 4 question 5 solution.

\subsection*{Answer}

Let there be $n$ symbols, and $n - 1$ operations between them. We introduce the
following symbols: \\
$T(i, j)$ represents the number of ways to parenthesize the 
symbols between i and j (both inclusive) such that the 
subexpression between i and j evaluates to true. \\  
$F(i, j)$ 
represents the number of ways to parenthesize the 
symbols between i and j (both inclusive) such that the 
subexpression between i and j evaluates to false. \\ 
So we need to solve $T(1, n)$, that is the the 
number of ways one can put parentheses
in the expression with $n$ symbols such that it
 will evaluate to true. To solve $T(1, n)$, we need to solve the two sub problems: 
 $T(i, j)$ and $F(i, j)$. \\
 The base case is 
  that $T(a, a)$ is 1 if symbol $a$ is true, and 0 if symbol $a$ is false. 
 The reverse applies to $F(a, a)$. Otherwise, we split sub problems around an operator $k$,
 so everything 
 in the left and right of the operator is in its own brackets. According to the truth table of
 `AND', `OR', `NAND', `NOR',
 we solve the sub problems in the following way 


 \[ T(i, j) = \sum_{k = i}^{j-1}  \begin{cases} 
       T(i, k)*T(k+1, j) & \\ \mbox{\qquad if operator $k$ is `AND' }\\
       T(i, k) * F(k + 1, j) + T(i, k) * T(k + 1, j) + F(i, k) * T(k + 1, j) & \\ 
       \mbox{\qquad if operator $k$ is `OR' }\\
       T(i, k) * F(k + 1, j) + F(i, k) * T(k + 1, j) + F(i, k) * F(k + 1, j) & \\ 
       \mbox{\qquad if operator $k$ is `NAND' }\\
       F(i, k) * F(k + 1, j) & \\ 
       \mbox{\qquad if operator $k$ is `NOR' }
    \end{cases}
 \]
 \[ F(i, j) = \sum_{k = i}^{j-1}  \begin{cases} 
       T(i, k) * F(k + 1, j) + F(i, k) * F(k + 1, j) + F(i, k) * T(k + 1, j) & \\ \mbox{\qquad if operator $k$ is `AND' }\\
       F(i, k) * F(k + 1, j) & \\ 
       \mbox{\qquad if operator $k$ is `OR' }\\
       T(i, k) * T(k + 1, j) & \\ 
       \mbox{\qquad if operator $k$ is `NAND' }\\
       T(i, k) * T(k + 1, j) + F(i, k) * T(k + 1, j) + T(i, k) * F(k + 1, j) & \\ 
       \mbox{\qquad if operator $k$ is `NOR' }
    \end{cases}
 \]
 After solving all sub problems, we can get the answer of the question from 
 $T(1,n)$.\\\\
 The time complexity is $O(n^3)$. There are $O(n^2)$ different ranges
 that $i$ and $j$ could cover, and each needs to evaluate
 $T(i, j)$ or $F(i, i)$ up to $n - 1$ different spilt points.
 So the time complexity is $O(n^2*n) = O(n^3)$



\end{document}
